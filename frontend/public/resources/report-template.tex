\documentclass[fontsize=10.5pt]{ctexart}

\special{dvipdfmx:config z 9}

\usepackage{array}
\usepackage[a4paper, left=3.18cm, right=3.18cm, top=2.54cm, bottom=2.54cm]{geometry}
\usepackage{minted}
\usepackage{tabularx}
\usepackage[most]{tcolorbox}
\usepackage{titlesec}
\usepackage{xcolor}
\usepackage{zhnumber}

\pagestyle{empty}
\renewcommand\arraystretch{1.5}
\titleformat{\section}{\fontsize{10.5}{10.5}\raggedright\bfseries}{\thesection、}{0em}{}
\titleformat{\subsection}{\fontsize{10.5}{10.5}\raggedright\bfseries}{(\thesubsection)}{0em}{}
\renewcommand\thesection{\zhnum{section}}
\renewcommand\thesubsection{\zhnum{subsection}}

\newtcolorbox{formal}[2][]{
    grow to right by=-5mm,
    grow to left by=-5mm,
    boxrule=0pt,
    boxsep=0pt,
    enhanced jigsaw,
    breakable,
    borderline west={3pt}{0pt}{gray}
}

\begin{document}

\begin{center}
    \textbf{\fontsize{14}{14}《现代密码学》实验报告}
\end{center}

\noindent
\begin{tabularx}{\textwidth}{|X|X|}
    \hline
    \textbf{实验名称:}(请自行修改)
    &
    \textbf{实验时间:}\today
    \\ \hline
    \textbf{学生姓名:}张三
    &
    \textbf{学号:}12345678
    \\ \hline
    \textbf{学生班级:}20网安/20保密管理
    &
    \textbf{成绩评定:}
    \\ \hline
\end{tabularx}

\section{实验目的}

通过实现……,理解……,提高……

\section{实验内容}

用 C/C++ 实现某个密码方案...

参数要求:...

\section{实验原理}

……

\section{实验步骤(源代码)}

规范的编程语言代码。

{
\linespread{1}
\renewcommand{\theFancyVerbLine}{\textcolor[HTML]{888888}{\arabic{FancyVerbLine}}}

\begin{minted}[
    style=colorful,
    xleftmargin=\parindent,
    fontsize=\small,
    linenos,
    breaklines,
    breaksymbol=\ensuremath{\textcolor[HTML]{888888}{\hookrightarrow}}
]{C}
#include <stdio.h>

// Code example

int main(int argc, char const *argv[]) {
    return 0;
}
\end{minted}
}

如果代码太长,可以分段展示重要部分。

\section{实验结果}

运行截图和最终结果。

\section{思考题}

\begin{formal}{}
    还有什么实现……的方法?
\end{formal}

……

\begin{formal}{}
    其他的问题?
\end{formal}

……

\section{实验总结}

……

\end{document}